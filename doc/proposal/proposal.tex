%% Project Proposal for BMB497F as taught during FA15 at Penn State.
%%
%% The document layout is based on the document originally found at:
%% prancer.physics.louisville.edu/classes/308/project_proposal/proposal.tex

\documentclass[dvips,12pt]{article}

\usepackage[pdftex]{graphicx}
\usepackage{url}
\usepackage{nth}

\setlength{\parindent}{0em}
\setlength{\parskip}{0.7em}

\begin{document}

\title{A De Bruijn Representation of a Repeat Family:\\Project Proposal}
\author{Sindhuja Parimalarangan, Matthew Allan, Scott Cheloha}
\date{October 12, 2015}

\maketitle

\section{Introduction}

De Bruijn graphs are used in computational biology to assemble
collections of short reads into genome sequences. This approach is
confounded by repetitive elements, which create cycles in the
graph. However, we believe we can actually take advantage of the
properties of De Bruijn graphs in order to represent a set of related
repeat elements and to derive a consensus (i.e. average) form of the
repeat element.

For our project, we will implement a De Bruijn graph representation of a
repeat family.

\section{Motivation}

While other representations of repeat families exist, there is plenty of
room for experimentation toward a better, perhaps standard,
representation.

For this project, as mentioned above, we will use De Bruijn graphs as a
basis for developing such a representation.  De Bruijn graphs are
presently used to assemble short reads.  We want to explore whether it
is possible to alter and/or augment the De Bruijn structure to
better represent repeat elements.

For instance, existing consensus sequences give the probability of
finding a base at a given position, but do not give the probability of
finding a given base in the position immediately \emph{after} said base.
De Bruijn graphs could show, e.g., that an Adenine at position 1 is
almost always followed by an Adenine at position 2, even if the
probabilities of finding an Adenine at positions 1 or 2 is only 10\% for
either position.

\section{Goals}

In order of decreasing priority, we want to:
\begin{enumerate}
\item Construct a De Bruijn graph representation of a chosen repeat
  family;
\item Develop a representation for a set of related repeat elements and
  derive a consensus (i.e. average) form for the repeat element; and
\item Design and implement an optimized algorithm for representing an
  input repeat family using De Bruijn graphs.
\end{enumerate}

\section{Resources}

The following subsections list the resources we anticipate will be
necessary to achieve the prenominated goals.  The lists will grow with
our understanding of the problem.

\subsection{Data}

At a minimum we will use the following datasets:
\begin{itemize}
\item GenBank
\item LINE/SINE element sequences of indeterminate source
\end{itemize}

\subsection{Hardware}

We will probably need time on a cluster of some capacity in order to
execute our software on larger datasets.

\subsection{Software}

Exactly what form the output of this project will take with regard to
implementation language, use of preexisting libraries, if any, etc., is
unclear.  For now, at least while researching for the project and
preparing the final report, we believe we will use the following:
\begin{itemize}
\item UCSC Genome Browser,
  for lots of things;
\item Galaxy,
  for data analysis;
\item RepeatMasker,
  as a reference implementation of some aspects of the project;
\item Velvet and related software,
  for sequence assembly;
\item RepeatFinder, RECON, and RepeatGluer,
  for finding and processing repeats;
\item Some sort of graph visualization library, for algorithm design and
  the final report; and
\item The GIMP, for manipulating those visualizations.
\end{itemize}

\section{Responsibilities}

Each team member will be partly responsible for determining how to use
De Bruijn graphs to represent a consensus sequence for our target repeat
families.  Each team member will also be partly responsible for deciding
upon which axes (computational efficiency, output accuracy, quality of
output representation, etc.) we will compare our results to those of
RepeatMasker.

In addition, each team member will be responsible for a more specific
partition of tasks tuned to their interests and abilities.

\textbf{Sindhuga Parimalarangan} will:
\begin{itemize}
\item Design and implement a software workflow for collecting, parsing,
  and storing data;
\item Optimize the algorithm vis a vis data structures and external
  libraries to mitigate redundant searching; and
\item Implement both the graph construction and consensus
  sequence-finding algorithms.
\end{itemize}

\textbf{Matthew Allan} will:
\begin{itemize}
\item Identify repeat target repeat families and document why they are
  suitable for the project;
\item Determine what RepeatMasker \emph{does} in order to determine a
  consensus sequence so that we something of a reference;
\item Run RepeatMasker as needed; and
\item Explain any differences between RepeatMasker's output and that of
  our implementation.
\end{itemize}

\textbf{Scott Cheloha} will:
\begin{itemize}
\item Scour the literature for De Bruijn graph construction algorithms;
\item Determine which of those algorithms exist in a form we can use
  (legally, architecturally, etc.) and which we must implement
  ourselves; and
\item Implement both the graph construction and consensus
  sequence-finding algorithms.
\end{itemize}


\section{Timeline}

On \textbf{November 9, 2015}, we will have:
\begin{enumerate}
\item Identified our target family/families of repeats;
\item Generated De Bruijn graphs for the same using some preexisting
  implementation;
\item Obtained any needed datasets and stored them (possibly on the
  cluster) in a suitable format;
\item Run RepeatMasker and learned what we can from it; and
\item Finalized the high-level details of the algorithm we intend to
  implement.
\end{enumerate}

On \textbf{December 12, 2015}, we will have:
\begin{enumerate}
\item A clean, optimized implementation of our algorithm;
\item The results of our algorithm in a legible format; and
\item Determined how (or how \emph{not}) to make a consensus sequence
  from De Bruijn graphs that is more informative than a table of letter
  frequencies.
\end{enumerate}

\end{document}
